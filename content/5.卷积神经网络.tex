\section{卷积神经网络}

\noindent\textbf{习题5-1} 1)证明公式(5.6)可以近似为离散信号序列x(t)关于t的二阶微分;2)对于二维卷积,设计一种滤波器来近似
实现对二维输入信号的二阶微分。

习题 5-1: 公式 (5.6):
\[
x''(t) = x(t+1) + x(t-1) - 2x(t) \tag{5.6}
\]

证明:

对于 \(x\):
\[
x'(t) = x(t+1) - x(t)
\]

\[
x''(t) = x'(t) - x'(t-1)
\]
\[
= [x(t+1) - x(t)] - [x(t) - x(t-1)]
\]
\[
= x(t+1) + x(t-1) - 2x(t)
\]

对于离散的函数来说,微分常常称为差分,在计算的时候又分为向前差分和向后差分。

对于二维的输入信号,由公式 (5.6) 得:
\[
\frac{\partial^2 f}{\partial x^2} = f(x+1, y) + f(x-1, y) - 2f(x, y)
\]
\[
\frac{\partial^2 f}{\partial y^2} = f(x, y+1) + f(x, y-1) - 2f(x, y)
\]

所以,二维的二阶微分为:
\[
\frac{\partial^2 f}{\partial x^2} + \frac{\partial^2 f}{\partial y^2} = f(x+1, y) + f(x-1, y) + f(x, y+1) + f(x, y-1) - 4f(x, y)
\]

由此我们可以设计滤波器:
\[
\begin{bmatrix}
0 & 1 & 0 \\
1 & -4 & 1 \\
0 & 1 & 0
\end{bmatrix}
\]

也称为 Laplace 算子。


\noindent\textbf{习题5-2} 证明宽卷积具有交换性,即公式(5.13).

\[
rot180(W) \tilde{\otimes} X
\]
\[
= rot180(W) \otimes \tilde{X}
\]
\[
= \tilde{X} \otimes rot180(W)
\]
\[
= X \tilde{\otimes} rot180(W)
\]
\[
= rot180(X) \tilde{\otimes} W
\]


\noindent\textbf{习题5-3} 分析卷积神经网络中用1X1的卷积核的作用。

\begin{enumerate}
\item 降维或升维,调整通道数
\item 降低计算复杂度,
\item 跨通道信息交互,其实就是不同通道之间的线性组合,这就是跨通道信息交互
\end{enumerate}

\noindent\textbf{习题5-4} 对于一个输入为100X100X256的特征映射组,使用3X3的卷积核,输出为100X100X256的特征
映射组的卷积层,求其时间和空间复杂度。如果引入一个1X1卷积核,先得到100X100X64的特征映射,再进行3X3的卷积,
得到100X100X256的特征映射,求其时间和空间复杂度。

(答案请自行查看插图,前两张为代码角度,后一张为直观角度)

\noindent\textbf{习题5-5} 对于一个二维卷积,输入为 $3 \times 3$,卷积核大小为 $2 \times 2$,试将卷积操作重写为仿射变换的形式。

解析:将 $3 \times 3$ 输入展开成 9 维向量。$3 \times 3$ 的输入,卷积核大小为 $2 \times 2$,一共做四次卷积操作。在对应的位置上填写卷积核的值。

\[
\begin{bmatrix}
w_{11} & w_{12} & 0 & 0 & 0 & 0 & 0 & 0 & 0 \\
w_{21} & w_{22} & 0 & 0 & 0 & 0 & 0 & 0 & 0 \\
0 & w_{11} & w_{12} & 0 & 0 & 0 & 0 & 0 & 0 \\
0 & w_{21} & w_{22} & 0 & 0 & 0 & 0 & 0 & 0 \\
0 & 0 & w_{11} & w_{12} & 0 & 0 & 0 & 0 & 0 \\
0 & 0 & w_{21} & w_{22} & 0 & 0 & 0 & 0 & 0 \\
0 & 0 & 0 & w_{11} & w_{12} & 0 & 0 & 0 & 0 \\
0 & 0 & 0 & w_{21} & w_{22} & 0 & 0 & 0 & 0 \\
0 & 0 & 0 & 0 & w_{11} & w_{12} & 0 & 0 & 0 \\
0 & 0 & 0 & 0 & w_{21} & w_{22} & 0 & 0 & 0 \\
0 & 0 & 0 & 0 & 0 & w_{11} & w_{12} & 0 & 0 \\
0 & 0 & 0 & 0 & 0 & w_{21} & w_{22} & 0 & 0 \\
0 & 0 & 0 & 0 & 0 & 0 & w_{11} & w_{12} & 0 \\
0 & 0 & 0 & 0 & 0 & 0 & w_{21} & w_{22} & 0 \\
0 & 0 & 0 & 0 & 0 & 0 & 0 & w_{11} & w_{12} \\
0 & 0 & 0 & 0 & 0 & 0 & 0 & w_{21} & w_{22} \\
\end{bmatrix}
\cdot
\begin{bmatrix}
x_1 \\ x_2 \\ x_3 \\ x_4 \\ x_5 \\ x_6 \\ x_7 \\ x_8 \\ x_9
\end{bmatrix}
\]

\noindent\textbf{习题5-6} 计算函数 $y = \max(x_1, \cdots, x_D)$ 和函数 $y = \arg\max(x_1, \cdots, x_D)$ 的梯度。

\section*{max 和 argmax}

\begin{itemize}
    \item $y = f(t)$ 是一般常见的函数式,如果给定一个值,$f(t)$ 函数式会赋一个值给 $y$。
    \item $y = \max f(t)$ 代表:$y$ 是 $f(t)$ 函数所有值中的值最大的 output。
    \item $y = \arg\max f(t)$ 代表:$y$ 是 $f(t)$ 函数中,会产生最大 output 的那个参数。
\end{itemize}

例如:

假设有一个函数 $f(t)$,$t$ 的可能范围是 $\{0, 1, 2\}$,$f(t=0) = 10$,$f(t=1) = 20$,$f(t=2) = 7$,那么分别对应的 $y$ 如下:

\[
y = \max f(t) = 20
\]
\[
y = \arg\max f(t) = 1
\]

argmax 是不导的,argmax \((x_1, x_2, \lambda): 
\begin{cases} 
0 & x_1 > x_2 \\ 
1 & x_1 < x_2 
\end{cases}\)
并且只要 $\(x_1\)\neq\(x_2\)$ ,那么对 $\(x_1\)$ 和 $\(x_2\)$ 进行一个很微小的变化,argmax 值是不发生改变的。因此这个时候 argmax 的梯度对于 $\(x_1\)$ 和 $\(x_2\)$ 来说都是 0。当 $\(x_1 = x_2\)$ 时,梯度值有一个会突然从 0 变为 1。

max \((x_1, x_2)\) 虽在 \(x_1 = x_2\)处不可导,但是其他点处,函数的值是 \(x_1\) 或 \(x_2\),并且 \(x_1\)、\(x_2\) 的微小移动是可以改变 max 的函数值的,因此对于 \(x_1\)、\(x_2\)的梯度要么是 \((0, 1)\),要么是 \((1, 0)\),可以看到梯度总是存在的,即训练最大的那个数对应的梯度。

这里的 $x_1$ 指的是 f_1(t)$,$x_d$ 指的是 $f_d(t)$


\noindent\textbf{习题5-7}  忽略激活函数,分析卷积网络中卷积层的前向计算和反向传播(公式 $(5.39)$)是一种转置关系。

解答:(请自行查看插图)

\noindent\textbf{习题5-8} 在空洞卷积中,当卷积核大小为 $K$,膨胀率为 $D$ 时,如何设置零填充 $P$ 的值以使得卷积为等宽卷积。

根据等宽卷积:

\[
\frac{(M - K' + 2P)}{S} + 1 = M, \quad \text{其中} S = 1,
\]

\[
K' = K + (K-1)(D-1),
\]
$
求得:

\[
P = \frac{(K-1)D}{2}.
\]



