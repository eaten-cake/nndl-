\section{卷积神经网络}

\noindent\textbf{习题5-1} 1)证明公式(5.6)可以近似为离散信号序列x(t)关于t的二阶微分;2)对于二维卷积,设计一种滤波器来近似
实现对二维输入信号的二阶微分。

\noindent\textbf{习题5-2} 证明宽卷积具有交换性,即公式(5.13).

\noindent\textbf{习题5-3} 分析卷积神经网络中用1X1的卷积核的作用。

\begin{enumerate}
\item 降维或升维,调整通道数
\item 降低计算复杂度,
\item 跨通道信息交互,其实就是不同通道之间的线性组合,这就是跨通道信息交互
\end{enumerate}

\noindent\textbf{习题5-4} 对于一个输入为100X100X256的特征映射组,使用3X3的卷积核,输出为100X100X256的特征
映射组的卷积层,求其时间和空间复杂度。如果引入一个1X1卷积核,先得到100X100X64的特征映射,再进行3X3的卷积,
得到100X100X256的特征映射,求其时间和空间复杂度。

