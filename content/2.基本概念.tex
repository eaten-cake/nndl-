\section{基本概念}

\noindent\textbf{习题2-1:} 分析为什么平方损失函数不适用于分类问题.

分类问题中的标签,是没有连续的概念的。每个标签之间的距离也是没有实际意义的,所以预测值和标签两个向量之间的平方差这个值不能反应分类这个问题的优化程度。

比如分类 1,2,3, 真实分类是1, 而被分类到2和3错误程度应该是一样的, 但是平方损失函数的损失却不相同.

\noindent\textbf{习题2-2:} 在线性回归中,如果我们给每个样本$(x^{(n)},y^{(n)})$赋予一个权重$r^{(n)}$,
经验风险函数为
\begin{equation}
\mathcal{R}(w) = \frac{1}{2}\sum_{n=1}^N r^{(n)}(y^{(n)} - w^T x^{(n)})^2
\end{equation}
计算其最优参数$w^{*}$,并分析权重$r^{(n)}$的作用。

我们令 $[r^{(n)}]^2 = r^{(n)}$ 则:

\[\mathcal{R}(w) = \frac{1}{2}\sum_{n=1}^N [r^{(n)}]^2(y^{(n)} - w^\top x^{(n)})^2\]

\[= \frac{1}{2}\sum_{n=1}^N \left(r^{(n)}(y^{(n)} - w^\top x^{(n)})\right)^2\]

\[= \frac{1}{2}\|r^\top(y - X^\top w)\|^2\]

对风险函数参数 $w$ 求导:

\[\frac{\partial}{\partial w}\mathcal{R}(w) = \frac{\partial}{\partial w}\frac{1}{2}\|r^\top(y - X^\top w)\|^2\]

\[= -Xrr^\top(y - X^\top w) = 0\]

\[w^* = (Xrr^\top X^\top)^{-1}Xrr^\top y\]

其中,参数 $r^{(n)}$ 是为了对不同的数据进行加权,相当于不同的数据对结果的影响程度会不同。如果有某个数据比较重要,希望对其能够给予高度重视,那么就可以设置相对较大的权重,反之则设置小一点。

