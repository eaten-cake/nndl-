\section{基本概念}

\noindent\textbf{习题2-1:} 分析为什么平方损失函数不适用于分类问题.

分类问题中的标签,是没有连续的概念的。每个标签之间的距离也是没有实际意义的,所以预测值和标签两个向量之间的平方差这个值不能反应分类这个问题的优化程度。

比如分类 1,2,3, 真实分类是1, 而被分类到2和3错误程度应该是一样的, 但是平方损失函数的损失却不相同.

\noindent\textbf{习题2-2:} 在线性回归中,如果我们给每个样本$(x^{(n)},y^{(n)})$赋予一个权重$r^{(n)}$,
经验风险函数为
\begin{equation}
\mathcal{R}(w) = \frac{1}{2}\sum_{n=1}^N r^{(n)}(y^{(n)} - w^T x^{(n)})^2
\end{equation}
计算其最优参数$w^{*}$,并分析权重$r^{(n)}$的作用。

我们令 $[r^{(n)}]^2 = r^{(n)}$ 则:

\[\mathcal{R}(w) = \frac{1}{2}\sum_{n=1}^N [r^{(n)}]^2(y^{(n)} - w^\top x^{(n)})^2\]

\[= \frac{1}{2}\sum_{n=1}^N \left(r^{(n)}(y^{(n)} - w^\top x^{(n)})\right)^2\]

\[= \frac{1}{2}\|r^\top(y - X^\top w)\|^2\]

对风险函数参数 $w$ 求导:

\[\frac{\partial}{\partial w}\mathcal{R}(w) = \frac{\partial}{\partial w}\frac{1}{2}\|r^\top(y - X^\top w)\|^2\]

\[= -Xrr^\top(y - X^\top w) = 0\]

\[w^* = (Xrr^\top X^\top)^{-1}Xrr^\top y\]

其中,参数 $r^{(n)}$ 是为了对不同的数据进行加权,相当于不同的数据对结果的影响程度会不同。如果有某个数据比较重要,希望对其能够给予高度重视,那么就可以设置相对较大的权重,反之则设置小一点。

\noindent\textbf{习题2-3} 证明在线性回归中,如果样本数量N小于特征数量D+1,则$XX^T$的秩最大为N。

1. 关于矩阵秩的性质:
- 由线性代数知识可知矩阵与其转置相乘的秩等于矩阵本身的秩,即 \[rank(XX^T) = rank(X)\]
- 矩阵 X 的秩必定满足条件:\[rank(X) \leq min(D+1, N)\],也就是说 X 的秩必须小于或等于 X 的行数和列数中的最小值
- 如果 $N < D+1$ 的话,那么必定有 \[rank(X) \leq N\]
- 故此,$XX^T$ 的秩最大为 $N$

2. 关于拟合问题:
- 如果特征数大于样本数的话,即使是常用的优化方法都无法去拟合数据,这种情况我们称之为拟合方程是欠定的
- 换句话说,这种情况实际就是线性方程组的未知数个数大于方程个数,不存在唯一非零解

\noindent\textbf{习题2-4}

\[R(\omega) = \frac{1}{2}\|y - X^T\omega\|^2 + \frac{1}{2}\lambda\|\omega\|^2\]

\[\frac{\partial R(\omega)}{\partial \omega} = \frac{1}{2}\frac{\partial \|y - X^T\omega\|^2}{\partial \omega} + \frac{1}{2}\lambda\frac{\partial \|\omega\|^2}{\partial \omega}\]

\[= -X(y - X^T\omega) + \lambda\omega\]

令$\frac{\partial R(\omega)}{\partial \omega}$为0:

\[Xy = (XX^T + \lambda I)\omega^*\]

\[\omega^* = (XX^T + \lambda I)^{-1}X\]

\noindent\textbf{习题2-5} 在线性回归中,若假设标签$y~N(\omega^Tx,\beta)$,并用最大似然估计来优化参数,验证最优参数
为公式(2.52)的解。

\[log p(y|X; \omega, \sigma) = \sum_{n=1}^N logN(y^{(n)}; \omega^T x^{(n)}, \sigma^2)\]

\[\frac{\partial log p(y|X; \omega, \sigma)}{\partial \omega} = \frac{\partial \sum_{n=1}^N \frac{(y^{(n)} - \omega^T x^{(n)})^2}{2\sigma^2}}{\partial \omega}\]

\[= \frac{\partial \frac{1}{2} \|y - X^T\omega\|^2}{\partial \omega}\]

\[= -X(y - X^T\omega)\]

令$\frac{\partial R(\omega)}{\partial \omega}$为0:

\[\omega^{ML} = (XX^T)^{-1}Xy\]

\noindent\textbf{习题2-6} 假设有N个样本$x^{(1)},x^{(2)},x^{(3)},...,x^{(N)}$服从正态分布$N(\mu,\delta^2)$,其中$\mu$未知
1)使用最大似然估计来求解最优参数$\mu^{ML}$;2)若参数$\mu$为随机变量,并服从正态分布$N(\mu_0,\delta^2_0)$,使用最大后验估计来求解
最优参数$\mu^{MAP}$。

1) 参数$\mu$在样本$\mathbf{x}$上的似然函数为

\[p(\mathbf{x}|\mu,\sigma^2)=\prod_{n=1}^N p(x^{(n)};\mu,\sigma^2),\]

对数似然函数为

\[\log p(\mathbf{x};\mu,\sigma^2)=\sum_{n=1}^N \log p(x^{(n)};\mu,\sigma^2)=\sum_{n=1}^N \left(\log\frac{1}{\sqrt{2\pi}\sigma}-\frac{(x^{(n)}-\mu)^2}{2\sigma^2}\right).\]

我们的目标是找到参数$\mu$的一个估计使得似然函数最大,等价于对数似然函数最大。令

\[\frac{\partial \log p(\mathbf{x};\mu,\sigma^2)}{\partial \mu}=\frac{1}{\sigma^2}\sum_{n=1}^N(x^{(n)}-\mu)=0,\]

得到$\mu^{ML}=\frac{1}{N}\sum_{n=1}^N x^{(n)}$,即样本均值。

2) 参数$\mu$的后验分布为

\[p(\mu|\mathbf{x};\mu_0,\sigma_0^2) \propto p(\mathbf{x}|\mu;\sigma^2)p(\mu;\mu_0,\sigma_0^2).\]

令似然函数$p(\mathbf{x}|\mu;\sigma^2)$为高斯密度函数,则后验分布的对数为

\[\log p(\mu|\mathbf{x};\mu_0,\sigma_0^2) \propto \log p(\mathbf{x}|\mu;\sigma^2)+\log p(\mu;\mu_0,\sigma_0^2)\]
\[\propto -\frac{1}{\sigma^2}\sum_{n=1}^N(x^{(n)}-\mu)^2-\frac{1}{\sigma_0^2}(\mu-\mu_0)^2.\]

令$\partial \log p(\mu|\mathbf{x};\mu_0,\sigma_0^2)/\partial \mu=0$,得到

\[\mu^{MAP}=\left(\frac{1}{\sigma^2}\sum_{n=1}^N x^{(n)}+\frac{\mu_0}{\sigma_0^2}\right)/\left(\frac{N}{\sigma^2}+\frac{1}{\sigma_0^2}\right).\]

证毕。

\noindent\textbf{习题2-7} 在习题2-6中,证明当$N \to \infty$时,最大后验估计趋向于最大似然估计。

证明:

\[\mu^{MAP} = \frac{\frac{\sum_{i=1}^N x_i}{\sigma^2} + \frac{\mu_0}{\sigma_0^2}}{\frac{N}{\sigma^2} + \frac{1}{\sigma_0^2}} = \frac{\sigma_0^2\sum_{i=1}^N x_i + \sigma^2\mu_0}{N\sigma_0^2 + \sigma^2}\]

\[\mu^{MLE} = \frac{1}{N}\sum_{i=1}^N x_i\]

当$N \to \infty$时,$\mu^{MAP}$近似等于$\mu^{MLE}$,因为其他数都为常数:

\[\lim_{N\to\infty} \mu^{MAP} = \lim_{N\to\infty} \frac{\sigma_0^2\sum_{i=1}^N x_i + \sigma^2\mu_0}{N\sigma_0^2 + \sigma^2} \approx \lim_{N\to\infty} \frac{\sigma_0^2\sum_{i=1}^N x_i}{N\sigma_0^2} = \frac{1}{N}\sum_{i=1}^N x_i = \mu^{MLE}\]

\noindent\textbf{习题2-8} 验证公式(2.61)

\noindent\textbf{习题2-9} 试分析什么因素会导致模型出现图2.6所示的高偏差和高方差情况。

拟合能力强的模型一般复杂度会比较高,容易过拟合,方差比较高。

如果限制模型复杂度,降低拟合能力,可能会欠拟合,偏差比较高.

\noindent\textbf{习题2-10} 验证公式(2.66)

\noindent\textbf{习题2-11} 分别用一元、二元和三元特征的词袋模型表示文本“我打了张三”和“张三打了我”,并分析不同模型的优缺点。

假设 $x_1$ 为"我打了张三" $x_2$ 为"张三打了我"

首先对句子进行分词:

\{"我","打了","张三"\}

1)一元特征:

共有特征:"我"、"张三"、"打了"

\[
x_1 = [1,1,1]
\]

\[
x_2 = [1,1,1]
\]

可以看出一元特征下无法表示语序,只要单词相同,向量便相同

2)二元特征:

共有特征:"\$我"、"\$张三"、"我打了"、"张三打了"、"打了张三"、"打了我"、"张三\#"、"我\#"

\[
x_1 = [1,0,1,0,1,0,1,0]
\]

\[
x_2 = [0,1,0,1,0,1,0,1]
\]

二元特征可以表示出每个单词相邻的顺序,不同语序其向量不同

3)三元特征:

共有特征:"\$我打了"、"\$张三打了"、"我打了张三"、"张三打了我"、"打了张三\#"、"打了我\#"

\[
x_1 = [1,0,1,0,1,0]
\]

\[
x_2 = [0,1,0,1,0,1]
\]

三元特征可以表示出单词前后两个相邻位置信息

\noindent\textbf{习题2-12} 对于一个三分类问题,数据集的真实标签和模型的预测标签如下:

\begin{tabular}{ccccccccccc}
\hline
真实标签 & 1 & 1 & 2 & 2 & 2 & 3 & 3 & 3 & 3 \\
预测标签 & 1 & 2 & 2 & 2 & 3 & 3 & 3 & 1 & 2 \\
\hline
\end{tabular}

分别计算模型的精确率、召回率、F1值以及它们的宏平均和微平均。

\textbf{解:} 设 $P_i$, $R_i$ 分别为类别 $i$ 的精确率和召回率

\[P_1 = \frac{1}{2} \qquad R_1 = \frac{1}{2} \qquad F_1^1 = \frac{2\cdot\frac{1}{2}\cdot\frac{1}{2}}{\frac{1}{2}+\frac{1}{2}} = \frac{1}{2}\]

\[P_2 = \frac{2}{4} = \frac{1}{2} \qquad R_2 = \frac{2}{3} \qquad F_2^1 = \frac{2\cdot\frac{1}{2}\cdot\frac{2}{3}}{\frac{1}{2}+\frac{2}{3}} = \frac{4}{7}\]

\[P_3 = \frac{2}{3} \qquad R_3 = \frac{2}{4} = \frac{1}{2} \qquad F_3^1 = \frac{2\cdot\frac{2}{3}\cdot\frac{1}{2}}{\frac{2}{3}+\frac{1}{2}} = \frac{4}{7}\]

宏平均:

\[P_{macro} = \frac{1}{3}(\frac{1}{2} + \frac{1}{2} + \frac{2}{3}) = \frac{5}{9}\]

\[R_{macro} = \frac{1}{3}(\frac{1}{2} + \frac{1}{2} + \frac{2}{3}) = \frac{5}{9}\]

\[F_1^{macro} = \frac{2\frac{5}{9}\frac{5}{9}}{\frac{5}{9} + \frac{5}{9}} = \frac{5}{9}\]

微平均:

\[P_{micro} = \frac{\frac{1+2+2}{3}}{\frac{2+4+3}{3}} = \frac{5}{9}\]

\[R_{micro} = \frac{\frac{1+2+2}{3}}{\frac{2+4+3}{3}} = \frac{5}{9}\]

\[F_1^{micro} = \frac{2\frac{5}{9}\frac{5}{9}}{\frac{5}{9} + \frac{5}{9}} = \frac{5}{9}\]

\[P_{micro} = R_{micro} = F_1^{micro}\]





